%----------------------------------------------------------------------------------------
%	PACKAGES AND THEMES
%----------------------------------------------------------------------------------------

\documentclass{beamer}
\usetheme{DarkConsole}


\usepackage{biblatex}
\addbibresource{references.bib}

%定义block颜色
\setbeamertemplate{blocks}[rounded=false,shadow=false]
\setbeamercolor{block title}{fg=white!90!blue,bg=black!10!blue}
\setbeamercolor{block body}{fg=white!90!blue,bg=black!10!blue}
\setbeamercolor*{block title example}{fg=white!90!green,bg=black!70!green}
\setbeamercolor*{block body example}{fg=white!90!green,bg=black!70!green}
\setbeamercolor*{block title alerted}{fg=white!90!red,bg=black!10!red}
\setbeamercolor*{block body alerted}{fg=white!90!red,bg=black!10!red}
% 设定中文字体
\usepackage[BoldFont,SlantFont,CJKchecksingle,CJKnumber]{xeCJK}
\usefonttheme[onlymath]{serif}

\xeCJKsetemboldenfactor{1}%只对随后定义的CJK字体有效
\setCJKfamilyfont{hei}{SimHei}
\xeCJKsetemboldenfactor{4}
\setCJKfamilyfont{song}{SimSun}
\xeCJKsetemboldenfactor{1}
\setCJKfamilyfont{fs}{FangSong}
\setCJKfamilyfont{kai}{KaiTi}
\setCJKfamilyfont{li}{LiSu}
\setCJKfamilyfont{xw}{STXinwei}

\setCJKmainfont[BoldFont={SimHei},ItalicFont={FangSong}]{SimSun}
\setCJKsansfont{SimSun}
\setCJKmonofont{FangSong}

\newcommand{\hei}{\CJKfamily{hei}}% 黑体   (Windows自带simhei.ttf)
\newcommand{\song}{\CJKfamily{song}}    % 宋体   (Windows自带simsun.ttf)
\newcommand{\fs}{\CJKfamily{fs}}        % 仿宋体 (Windows自带simfs.ttf)
\newcommand{\kaishu}{\CJKfamily{kai}}      % 楷体   (Windows自带simkai.ttf)
\newcommand{\li}{\CJKfamily{li}}        % 隶书   (Windows自带simli.ttf)
\newcommand{\xw}{\CJKfamily{xw}}        % 隶书   (Windows自带simli.ttf)
\newfontfamily\arial{Arial}
\newfontfamily\timesnewroman{Times New Roman}



\defaultfontfeatures{Mapping=tex-text}
\usepackage{xunicode}
\usepackage{xltxtra}
\XeTeXlinebreaklocale "zh"
\XeTeXlinebreakskip = 0pt plus 1pt minus 0.1pt
 
\usepackage{colortbl,xcolor}
\usepackage{hyperref}
\hypersetup{xetex,bookmarksnumbered=true,bookmarksopen=true,pdfborder=1,breaklinks, colorlinks, linkcolor=, urlcolor=structure.fg}
 
\usepackage{graphicx} % Allows including images
\usepackage{booktabs} % Allows the use of \toprule, \midrule and \bottomrule in tables
\usepackage{mathtools} %数学公式中case情况
\usepackage{amsthm}
\usepackage{amsfonts}
\usepackage{amsmath,amssymb, mathrsfs, bm}
\usepackage{tikz}% tikz做图
\usetikzlibrary{snakes}
\usepackage{minted}%code highlighting

%设定目录
\AtBeginSection[]%自动加入目录
{
\begin{frame}<beamer>
\frametitle{目录}
\tableofcontents[currentsection]
\end{frame}
}
\setbeamertemplate{section in toc}[sections numbered]
\setbeamertemplate{subsection in toc}[subsections numbered]
\defbeamertemplate{subsection page}{mine}[1][]{%
  \begin{centering}
    {\usebeamerfont{subsection name}\usebeamercolor[fg]{subsection name}#1}
    \vskip1em\par
    \begin{beamercolorbox}[sep=8pt,center,#1]{part title}
      \usebeamerfont{subsection title}\thesection.\thesubsection~\insertsubsection\par
    \end{beamercolorbox}
  \end{centering}
}
\setbeamertemplate{subsection page}[mine]
\AtBeginSubsection{\frame{\subsectionpage}}%自动加入标题页面

%引用文献高亮
\makeatletter
\let\@mycite\@cite
\def\@cite#1#2{{\hypersetup{linkcolor=blue!60!black}[{#1\if@tempswa , #2\fi}]}}
\makeatother

%中文图、表
\renewcommand{\figurename}{图}
\renewcommand{\tablename}{表}
\setbeamertemplate{caption}[numbered] %图表编号

%mindmap
\usetikzlibrary[topaths,mindmap,backgrounds]
\usetikzlibrary{chains,decorations.pathmorphing,positioning,fit}
\usetikzlibrary{decorations.shapes,calc,}
\usetikzlibrary{decorations.text,matrix}
\usetikzlibrary{arrows,shapes.geometric,shapes.symbols,scopes}
% A counter, since TikZ is not clever enough (yet) to handle
% arbitrary angle systems.
\newcount\mycount
\tikzset{
    invisible/.style={opacity=0},
    visible on/.style={alt=#1{}{invisible}},
    alt/.code args={<#1>#2#3}{%
    \alt<#1>{\pgfkeysalso{#2}}{\pgfkeysalso{#3}}
  },
}

\tikzset{
  orp/.style={
    overlay,
    remember picture,
  },
}
\newcommand<>\fadetitle[2][]{%
    \begin{tikzpicture}
      [
        orp,
        fade title fill/.style={
          fill=black,
          opacity=0.6,
        },
        #1
      ]
      \path[fade title fill] (current page.south west) +(-1cm, -1cm) rectangle ($ (current page.north east) + (2cm, 2cm) $);
      \node[above] at (current page) {\textbf{#2}};
    \end{tikzpicture}%
}
\tikzstyle{nodetype1pre}= [circle, fill=black!60]
\tikzstyle{nodetype1}= [circle, fill=structure.fg!30]
\tikzstyle{unnodetype1}= [circle, fill=black!60]
\tikzstyle{linktype1pre}= [-latex]
\tikzstyle{linktype1}= [-latex, draw=purple, line width=1.5pt]
\tikzstyle{dilinktype1pre}= [-]
\tikzstyle{dilinktype1}= [-, draw=purple, line width=1.5pt]
\tikzstyle{unlinktype1}= [-latex]
\tikzstyle{linktype2}= [-latex, snake=snake,line after snake=4mm, draw=structure.fg, line width=1.5pt]
\tikzstyle{every picture}+=[remember picture]


\tikzstyle{nodeisprior}= [circle, fill=blue!70!black, draw=white]
\tikzstyle{nodeislatent}= [circle, fill=green!40!black, draw=white]

\addtobeamertemplate{block begin}{%
  \setlength{\textwidth}{\textwidth}%
}{}

\newcommand{\point}[1]{\textbf{\usebeamercolor[fg]{structure}\hei#1}} %定义高亮文本
\newcommand{\bcite}[1]{\textbf{\textcolor{red!80}{\textsuperscript{[}\footfullcite{#1}\textsuperscript{]}}}}

%----------------------------------------------------------------------------------------
%	TITLE PAGE
%----------------------------------------------------------------------------------------

\title[亚克隆重组]{基于测序数据的肿瘤亚克隆重组} % The short title appears at the bottom of every slide, the full title is only on the title page

\author{初砚硕} % Your name
\institute[生物信息]{计算机学院,生物信息}
\date{\today} % Date, can be changed to a custom date

\begin{document}

\begin{frame}
\titlepage 
\end{frame}

%\begin{frame}
%\vspace*{12em}
%\begin{itemize}
%  \item  \tiny{本作品采用知识共享 署名-非商业性使用-相同方式共享 3.0 中国大陆 许可协议进行许可。要查看该许可协议,可访问 \url{http://creativecommons.org/licenses/by-nc-sa/3.0/cn/}}
%\end{itemize}
%\end{frame}

\begin{frame}
\frametitle{三个“?”}
\begin{block}{为什么肿瘤/癌症难以治疗?}
  肿瘤是由多种细胞组成的复杂混合体,大致包括发生变异的肿瘤细胞、未发生变异的肿瘤细胞、正常细胞(这种现象称为\point{异质性})。
一种治疗方法,只能针对个别成分发生作用,未被杀死的肿瘤细胞会迅速占据被杀死细胞的空间和营养,导致病情越治越坏。
\end{block}
\begin{block}{什么是亚克隆?}
目前没有一个标准的定义\bcite{gawad2016single}。
肿瘤中的细胞按照基因组相似度聚类(类别数未知),每一个类就是亚克隆。
\end{block}
\begin{block}{亚克隆重组有什么意义?}
  获得肿瘤的组成成分有利于制定更清晰的肿瘤治疗方案。
  获得肿瘤组成成分的过程称为\point{亚克隆重组}。
\end{block}
\end{frame}

\bgroup
\setbeamercolor{background canvas}{bg=white}
\setbeamercolor{frametitle}{fg=black}
\setbeamercolor{normal text}{fg=black}
\begin{frame}
\frametitle{肿瘤的异质性是进化过程}
\begin{figure}
\includegraphics[width=0.9\linewidth]{pic/hetero.pdf}
\end{figure}
\end{frame}
\egroup


\begin{frame}
\frametitle{目录}
\tableofcontents
\end{frame}

%----------------------------------------------------------------------------------------
%	PRESENTATION SLIDES
%----------------------------------------------------------------------------------------
%------------------------------------------------

\section{肿瘤/正常配对样本的全基因组测序数据的二维GC bias及其校正方法}

\begin{frame}
\frametitle{肿瘤/正常样本的测序数据}
\begin{figure}
\includegraphics[width=0.9\linewidth]{pic/IGV.png}
\caption{肿瘤和正常样本配对测序(Tumor nomral paired sequencing)}
\end{figure}
\pause
\fadetitle{\Huge{\centerline{如何检测肿瘤中的变异?}}}
\end{frame}

\subsection{SCNA检测流程}
\begin{frame}
\frametitle{SCNA检测流程}
  \begin{itemize}
    \item 根据覆盖度片段  Segmentation (BIC-seq)
    \item 根据每一个片段 (segment)之内的read性质,确定SCNA类型
    \begin{itemize}
      \item 覆盖度 read depth/count
      \item 等位型 allele type
    \end{itemize}
  \end{itemize}
  \pause
      \begin{block}{影响覆盖度的因素}
    \begin{itemize}
      \item 平均拷贝数 Average copy number
      \item Mappability (GC bias)
    \end{itemize}
      \end{block}
\end{frame}

\begin{frame}
现有的对影响覆盖度的因素建模的方法如下\bcite{li2015mixclone},使用$\theta_j$表示第$j$个片段内的由片段长度和映射性(mappability)造成的不均匀性,$\bar{C}_j$表示该片段的平均拷贝数,$D_j^N$表示在正常样本中的这个片段内的read数, $D_j^T$表示在肿瘤样本中的这个片段内的read数, 那么对于片段$i$和片段$j$,有${D_i^T}/{D_i^T} ={\bar{C}_i\theta_i}/{\bar{C}_j\theta_j}$,其中$\theta_i / \theta_j = D_i^N / D_j^N$ .
如果
\begin{equation}
    \frac{D_i^T}{D_j^T} = \frac{\bar{C}_i\theta_i}{\bar{C}_j\theta_j} = \frac{\bar{C}_i}{\bar{C}_j} * \frac{D_i^N}{D_j^N},
\end{equation}
那么,
\begin{equation}
  \log{\frac{D_i^T}{D_i^N}} - \log{\frac{D_j^T}{D_j^N}} = \log{\frac{\bar{C}_i}{\bar{C}_j}}.
  \label{eqassumption_mixclone}
\end{equation}
\pause
\fadetitle{公式~(\ref{eqassumption_mixclone})说明,对于所有的片段$i = 1, \ldots, m$,$\log{\frac{D_i^T}{D_i^N}}$可以进行近邻聚类。}
\end{frame}

\begin{frame}
\begin{figure}
\includegraphics[width=0.6\linewidth]{pic/stripe.pdf}
\caption{Stripe}
\end{figure}
\end{frame}

\subsection{Pre-SCNAClonal --- 肿瘤/正常样本全基因组测序数据的一种二维GC-bias校正和可视化工具}

\section{亚克隆重组模型}

\subsection{VAF信号校正}

\begin{frame}
\begin{figure}
\includegraphics[width=0.8\linewidth]{pic/BAFrd.pdf}
\caption{BAF read count distribution}
\end{figure}
\end{frame}

\begin{frame}
\frametitle{VAF信号过滤(平滑)}
\begin{figure}
\includegraphics[width=\linewidth]{pic/BAFdist.pdf}
\caption{BAF distribution}
\end{figure}
\end{frame}

\subsection{贝叶斯概率模型}

\begin{frame}
\begin{block}{如何利用(条带的)先验信息?}
\begin{itemize}
\item 平均拷贝数相等
\item 基线拷贝数为2
\item 其余条带的平均拷贝数由下到上递增
\end{itemize}
\end{block}
\end{frame}


%\begin{frame}
%  \begin{figure}
%  \begin{center}
%    \tikzstyle{scalestyle}= [scale=1]
%  \begin{tikzpicture}[scale=1.4]
%	\foreach \symbol/\name/\x/\y in {$t_m$/0/0/0, $b_m^n$/1/0/1, $H_0$/5/-1/4, {\tiny${ a_\alpha, b_\alpha}$}/6/1/5, $d_m$/7/-1/1, $\pi_m^n$/9/2/1, {\tiny${a_s, b_s}$}/11/2/5}
%	{
%	  \node[nodeisprior, scalestyle] (N-\name) at (\x, \y) {\symbol};
%	}
%	\foreach \symbol/\name/\x/\y in { $\phi^n$/2/0/2, $H$/3/0/3, $\alpha$/4/1/4, $\psi_m^n$/8/1/1, ${s}$/10/2/4}
%	{
%	  \node[nodeislatent, scalestyle] (N-\name) at (\x, \y) {\symbol};
%	}
%	\foreach \from/\to in {6/4, 4/3, 5/3, 3/2, 2/1, 0/1, 7/1, 8/1, 9/8, 10/1, 11/10}
%	{
%	  \draw[unlinktype1] (N-\from) -- (N-\to);
%	}
%
%  \end{tikzpicture}
%  \end{center}
%\caption{PyClone 的概率模型}
%  \end{figure}
%\end{frame}

%\begin{frame}
%  \begin{figure}
%  \begin{center}
%    \tikzstyle{scalestyle}= [scale=1]
%  \begin{tikzpicture}[yscale=1.6, xscale=2.5]
%	\foreach \symbol/\name/\x/\y in {$b_{i,m}^n$/1/0/1, $H_0$/5/-1/4, \tiny${a_\alpha, b_\alpha}$/6/1/5, $d_{i,m}^n$/7/-1/1, \tiny${{D_i^{T/N}}^n}$/11/1/2, $S^n$/13/2/4}
%	{
%	  \node[nodeisprior, scalestyle] (N-\name) at (\x, \y) {\symbol};
%	}
%	\foreach \symbol/\name/\x/\y in { $\phi^n$/2/0/2, $H$/3/0/3, $\alpha$/4/1/4, $\psi_{i,m}^n$/8/1/1,$\pi_{i, m}^n$/9/2/1, $C_{s,i}^n$/10/2/2, $\beta_s^n$/12/2/3}
%	{
%	  \node[nodeislatent, scalestyle] (N-\name) at (\x, \y) {\symbol};
%	}
%	\foreach \from/\to in {6/4, 4/3, 5/3, 3/2, 2/1, 7/1, 8/1, 9/8, 10/9, 10/11, 2/11, 12/10, 13/12}
%	{
%	  \draw[unlinktype1] (N-\from) -- (N-\to);
%	}
%
%  \end{tikzpicture}
%  \end{center}
%\caption{概率模型图}
%  \end{figure}
%\end{frame}

\begin{frame}
\frametitle{概率模型}
\begin{columns}[c] % The "c" option specifies centered vertical alignment while the "t" option is used for top vertical alignment
\column{.45\textwidth} % Left column and width
\vspace*{-1cm}
  \begin{figure}
  \begin{center}
    \tikzstyle{scalestyle}=[scale=1]
  \begin{tikzpicture}[yscale=1.6, xscale=1.5]
    \foreach \symbol/\name/\x/\y in {$b_{i,m}^n$/1/0/1, $H_0$/5/-1/3, \tiny${a_\alpha, b_\alpha}$/6/0/5, $d_{i,m}^n$/7/-1/1, \tiny${{D_{s,i}^{T}}^n}$/11/1/3, $\bm{\beta}_s^n$/13/2/4, \tiny{${{D_{s,i}^{N}}^n}$}/14/1/4}
	{
	  \node[nodeisprior, scalestyle] (N-\name) at (\x, \y) {\symbol};
	}
	\foreach \symbol/\name/\x/\y in { $\phi^n$/2/0/2, $H$/3/0/3, $\alpha$/4/0/4, $\psi_{i,m}^n$/8/1/1,$\pi_{i, m}^n$/9/2/1, $C_{s,i}^n$/10/2/2, $\bm{L}_s^n$/12/2/3}
	{
	  \node[nodeislatent, scalestyle] (N-\name) at (\x, \y) {\symbol};
	}
	\foreach \from/\to in {6/4, 4/3, 5/3, 3/2, 2/1, 7/1, 8/1, 9/8, 10/9, 10/11, 2/11, 12/10, 13/12,14/11, 2/10}
	{
	  \draw[unlinktype1] (N-\from) -- (N-\to);
	}

  \end{tikzpicture}
  \end{center}
  \end{figure}
\column{.45\textwidth} % Left column and width
$$\alpha \sim \text{Gamma}(a_\alpha, b_\alpha)$$
$$H_0 \sim \text{Uniform}([0,1]^M)$$
$$H \sim \text{DP}(H_0, \alpha)$$
$$\phi^n  \sim H$$
$$C \sim \text{Categorical}(\bm{L}_s^n)$$
$$\bm{L}_s^n \sim \text{Dir}(\bm{\beta}_s^n)$$
\end{columns}
\end{frame}

\begin{frame}
\frametitle{概率模型}
\begin{columns}[c] % The "c" option specifies centered vertical alignment while the "t" option is used for top vertical alignment
\column{.45\textwidth} % Left column and width
\vspace*{-1cm}
  \begin{figure}
  \begin{center}
    \tikzstyle{scalestyle}=[scale=1]
  \begin{tikzpicture}[yscale=1.6, xscale=1.5]
    \foreach \symbol/\name/\x/\y in {$b_{i,m}^n$/1/0/1, $H_0$/5/-1/3, \tiny${a_\alpha, b_\alpha}$/6/0/5, $d_{i,m}^n$/7/-1/1, \tiny${{D_{s,i}^{T}}^n}$/11/1/3, $\bm{\beta}_s^n$/13/2/4, \tiny{${{D_{s,i}^{N}}^n}$}/14/1/4}
	{
	  \node[nodeisprior, scalestyle] (N-\name) at (\x, \y) {\symbol};
	}
	\foreach \symbol/\name/\x/\y in { $\phi^n$/2/0/2, $H$/3/0/3, $\alpha$/4/0/4, $\psi_{i,m}^n$/8/1/1,$\pi_{i, m}^n$/9/2/1, $C_{s,i}^n$/10/2/2, $\bm{L}_s^n$/12/2/3}
	{
	  \node[nodeislatent, scalestyle] (N-\name) at (\x, \y) {\symbol};
	}
	\foreach \from/\to in {6/4, 4/3, 5/3, 3/2, 2/1, 7/1, 8/1, 9/8, 10/9, 10/11, 2/11, 12/10, 13/12,14/11, 2/10}
	{
	  \draw[unlinktype1] (N-\from) -- (N-\to);
	}

  \end{tikzpicture}
  \end{center}
  \end{figure}
\column{.45\textwidth} % Left column and width
$$\psi_{i, m}^n | \pi_{i, m}^n \sim \text{Categorical}(\pi_{i,m}^n)$$
$$\pi_{i, m}^n | C_{s, i}^n \sim \text{Categorical}(C_{s,i}^n)$$
\begin{equation*}
\begin{split}
b_{i, m}^n &| d_{i, m}^n, \psi_{i,m}^n, \phi^n  \sim \\
&\text{Binomial}(d_{i,m}^n, \xi(\phi^n, \psi_{i,m}^n)) \\
\end{split}
\end{equation*}
$$\xi = \frac{\phi^n * C * \mu + (1- \phi^n) *2*\frac{1}{2}}{\phi^n * C + (1- \phi^n) *2}$$
%\noindent\makebox[\linewidth]{\usebeamercolor[fg]{structure}\rule{\linewidth}{0.8pt}}
${D_{s, i}^T}^n \sim \text{Possion}\left(\frac{\overline{C}_{s,i}^n}{2} * \sqrt[J]{\prod\limits_{j = 1}^J\frac{D_{j}^T}{D_j^N}} * {D_{s, i}^N}^n\right)$
\end{columns}
\end{frame}

\begin{frame}
\frametitle{概率模型参数}
$$
\pi_{i,m}^n \in \{\oslash, \text{P}, \text{M}, \text{PP}, \text{PM}, \text{MM}, \text{PPP}, \ldots, \text{MMMMMMM}\}
$$
$$
C_{s, i}^n \in \{0, 1, 2, \ldots, 7\}
$$
$$
\bm{L}_s^n = \{l_0, l_1, \ldots, l_7\}
$$
$$
\bm{\beta}_s^n = \{\beta_{s0}^n, \beta_{s1}^n,\ldots, \beta_{s7}^n\}
$$
\begin{equation*}
\begin{array}{rl}
  j &= 0, \ldots, 7 \\
  \beta_{sj}^n & =\left\{
\begin{array}{l}
  \frac{0.8}{m},\; \text{if}\; j = s \\
  \frac{0.2}{8 - m},\; \text{if}\; j \neq s\\
\end{array}
\right.
\end{array}
\end{equation*}
\end{frame}

\begin{frame}
  \frametitle{抽样方法}
  \begin{equation*}
\begin{split}
  P(\phi, C | D, b) & \varpropto P(D, b|\phi, C) * P(\phi) * P(C|\phi) \\
		    & \varpropto P(D | \phi, C) * P(b | \phi,C) * P(\phi) * P(C|\phi)  \\
      & = P(\phi) * P(C | \phi) * \\
\end{split}
  \end{equation*}
  \begin{equation*}
      \prod\limits_{n = 1} ^N \prod\limits_{s=1}^S \prod\limits_{i =1} ^I \left[P\left( \left\{ D_{s,j}^T\right\}^n | \left\{ D_{s,j}^N\right\}^n, \phi^n, C_{s,i}^n\right)\prod\limits_{m=1}^M P\left( b_{i,m}^n | d_{i,m}^n, \phi^n, \psi_{i,m}^n \right)  \right]
  \end{equation*}
\end{frame}


\end{document}
\begin{frame}
  \begin{block}{扔硬币问题}
   扔一次硬币,正面朝上,再扔一次还是正面的概率(硬币均匀度)是多少? 
  \end{block}
  \begin{block}{贝叶斯公式}
 \begin{equation*}
 \begin{split}P(x | D )& = \frac{P(x)\times P(D | x)}{P(D)}\\
		       & = \frac{P(x)\times P(D | x)}{\int_0^1P(x)\times P(D | x)dx}\\
 \end{split}
 \end{equation*}
  \end{block}
\end{frame}

\begin{frame}
“别噎死”门徒们认为硬币均匀程度是个概率分布,先验设为均匀分布,借用该文中的定义,应该是$P(p_\text{正}) \sim \mathrm{Uniform}(0,1)$。 那么扔了一次正面朝上之后,这个均匀程度被修正了,这个均匀程度的修正过程就是“别噎死”公式:

\begin{equation*}
  \begin{split}
P(p_\text{正} | m_\text{正}=1, m_\text{负}=0)& = \frac{p_\text{正}^1(1-p_\text{正})^0P(p_\text{正})}{\int_0^1 t^1(1-t)^0dt}\\
&= \frac{\Gamma(3)}{\Gamma(2)\Gamma(1)} p_\text{正}\\
\end{split}
\end{equation*}
右侧正好对应的是$p_\text{正}$的贝塔分布:
\begin{equation*}
P(p_\text{正} | m_\text{正}, m_\text{负}) \sim \mathrm{Be}(m_\text{正}+1, m_\text{负}+1)
\end{equation*}
\end{frame}

\begin{frame}
\begin{block}{还是硬币问题}
有一对夫妇已先生了一个儿子,之后又生了一个儿子,那么这对夫妇再生一个小孩是儿子的概率是多少?
\end{block}
第一步可以得到:
\begin{equation*}
P(p_{\text{正}_1} | m_\text{正}=1, m_\text{负}=0) \sim \mathrm{Be}(2,1)
\end{equation*}
再投一次硬币:
\begin{equation*}
  \begin{split} P(p_{\text{正}_2} | m_\text{正}=1, m_\text{负}=0) &= \frac{2p_{æ­£}\times p_{正}}{\int_0^1 2t^2dt}\\& = \frac{\Gamma(4)}{\Gamma(3)\Gamma(1)}p_{正}^2
\end{split}
\end{equation*}
即:
\begin{equation*}
  P(p_{\text{正}_2} | m_\text{正}=1, m_\text{负}=0) = \mathrm{Be}(3,1) 
\end{equation*}
\end{frame}

%\subsection{中英文混合排版}
%\begin{frame}
%\frametitle{Paragraphs of Text}
%中英文混合排版,中英文混合排版,中英文混合排版,中英文混合排版,中英文混合排版,中英文混合排版,中英文混合排版,中英文混合排版,中英文混合排版,\point{中英文混合排版},Sed iaculis dapibus gravida. Morbi sed tortor erat, nec interdum arcu. Sed id lorem lectus. Quisque viverra augue id sem ornare non aliquam nibh tristique. Aenean in ligula nisl. Nulla sed tellus ipsum. Donec vestibulum ligula non lorem vulputate fermentum accumsan neque mollis.
%\end{frame}
%
%%------------------------------------------------
%
%\begin{frame}
%\frametitle{Bullet Points}
%\begin{itemize}
%\item Lorem ipsum dolor sit amet, consectetur adipiscing elit
%\item Aliquam blandit \tikz[baseline,inner sep=0] \node[anchor=base] (n1) {paths};faucibus nisi, sit amet dapibus enim tempus eu
%\item Nulla commodo, erat quis gravida posuere, elit lacus lobortis est, quis porttitor odio mauris at libero
%\item Nam cursus est eget velit posuere pellentesque
%\item Vestibulum faucibus velit a augue condimentum quis \tikz[baseline,inner sep=0] \node[anchor=base] (n2) {convallis };nulla gravida
%\end{itemize}
%\pause
%\tikz[overlay]\draw[thick,green,->] (n2) -- (n1);
%\end{frame}
%
%%------------------------------------------------
%
%\begin{frame}
%\frametitle{Blocks of Highlighted Text}
%\begin{block}{普通框}
%Lorem ipsum dolor sit amet, consectetur adipiscing elit. Integer lectus nisl, ultricies in feugiat rutrum, porttitor sit amet augue. Aliquam ut tortor mauris. Sed volutpat ante purus, quis accumsan dolor.
%\end{block}
%
%\begin{exampleblock}{举例框}
%Pellentesque sed tellus purus. Class aptent taciti sociosqu ad litora torquent per conubia nostra, per inceptos himenaeos. Vestibulum quis magna at risus dictum tempor eu vitae velit.
%\end{exampleblock}
%
%\begin{alertblock}{警告框}
%Suspendisse tincidunt sagittis gravida. Curabitur condimentum, enim sed venenatis rutrum, ipsum neque consectetur orci, sed blandit justo nisi ac lacus.
%\end{alertblock}
%\end{frame}
%
%%------------------------------------------------
%
%\begin{frame}
%\frametitle{Multiple Columns}
%\begin{columns}[c] % The "c" option specifies centered vertical alignment while the "t" option is used for top vertical alignment
%
%\column{.45\textwidth} % Left column and width
%\textbf{Heading}
%\begin{enumerate}[<+->]
%\item Statement
%\item Explanation
%\item Example
%\end{enumerate}
%
%\column{.5\textwidth} % Right column and width
%Lorem ipsum dolor sit amet, consectetur adipiscing elit. Integer lectus nisl, ultricies in feugiat rutrum, porttitor sit amet augue. Aliquam ut tortor mauris. Sed volutpat ante purus, quis accumsan dolor.
%
%\end{columns}
%\end{frame}
%
%
%\section{图、表、公式} 
%\subsection{普通插图} %a subsection can be created just before a set of slides with a common theme to further break down your presentation into chunks
%\begin{frame}
%\frametitle{Figure}
%Uncomment the {code} on this slide to include your own image from the same {directory} as the template .TeX file.
%\begin{figure}
%\includegraphics[width=0.2\linewidth]{pic/1.jpg}
%\caption{example}
%\end{figure}
%\end{frame}
%
%\subsection{tikz绘图} %a subsection can be created just before a set of slides with a common theme to further break down your presentation into chunks
%\begin{frame}
%\begin{tikzpicture}[scale=2,cap=round]
%  % Local definitions
%  \def\costhirty{0.8660256}
%
%  % Colors
%  \colorlet{anglecolor}{green!50!black}
%  \colorlet{sincolor}{red}
%  \colorlet{tancolor}{orange!80!black}
%  \colorlet{coscolor}{blue}
%
%  % Styles
%  \tikzstyle{axes}=[]
%  \tikzstyle{important line}=[very thick]
%  \tikzstyle{information text}=[rounded corners,fill=red!10,inner sep=1ex]
%
%  % The graphic
%  \draw[style=help lines,step=0.5cm] (-1.4,-1.4) grid (1.4,1.4);
%
%  \draw (0,0) circle (1cm);
%
%  \begin{scope}[style=axes]
%    \draw[->] (-1.5,0) -- (1.5,0) node[right] {$x$};
%    \draw[->] (0,-1.5) -- (0,1.5) node[above] {$y$};
%
%    \foreach \x/\xtext in {-1, -.5/-\frac{1}{2}, 1}
%      \draw[xshift=\x cm] (0pt,1pt) -- (0pt,-1pt) node[below,fill=black]
%            {$\xtext$};
%
%    \foreach \y/\ytext in {-1, -.5/-\frac{1}{2}, .5/\frac{1}{2}, 1}
%      \draw[yshift=\y cm] (1pt,0pt) -- (-1pt,0pt) node[left,fill=black]
%            {$\ytext$};
%  \end{scope}
%
%  \filldraw[fill=green!20,draw=anglecolor] (0,0) -- (3mm,0pt) arc(0:30:3mm);
%  \draw (15:2mm) node[anglecolor] {$\alpha$};
%
%  \draw[style=important line,sincolor]
%    (30:1cm) -- node[left=1pt,fill=black] {$\sin \alpha$} +(0,-.5);
%
%  \draw[style=important line,coscolor]
%    (0,0) -- node[below=2pt,fill=black] {$\cos \alpha$} (\costhirty,0);
%
%  \draw[style=important line,tancolor] (1,0) --
%    node [right=1pt,fill=black]
%    {
%      $\displaystyle \tan \alpha \color{black}=
%      \frac{{\color{sincolor}\sin \alpha}}{\color{coscolor}\cos \alpha}$
%    } (intersection of 0,0--30:1cm and 1,0--1,1) coordinate (t);
%
%  \draw (0,0) -- (t);
%
%  \end{tikzpicture}
%  \frametitle{tikz picture sample}
%\end{frame}
%
%\begin{frame}
%\frametitle{手绘图}
%\tikzstyle{every picture}+=[remember picture]
%\[  y =  \tikz[baseline]{\node[fill=blue!50,anchor=base] (t1){$a$};} x +
%\tikz[baseline]{\node[fill=red!50,anchor=base ] (t2){$b$};}
%\]
%\begin{itemize}
%\item \tikz\node [fill=blue!50,draw,circle] (n1) {};\ slope
%\item \tikz\node [fill=red!50,draw,circle] (n2) {};\ y-intercept
%\end{itemize}
%\begin{tikzpicture}[overlay,>=latex]
%\path[blue,->] (n1.north) edge [out= 60, in= 135] (t1.north west);
%\path[red,->] (n2.south) edge [out=-70, in=-110] (t2.south);
%\end{tikzpicture}
%\end{frame}
%%------------------------------------------------
%
%%------------------------------------------------
%\subsection{公式和列表} 
%
%\begin{frame}
%\frametitle{公式说明}
%\begin{equation}
%  \frac{\mathrm{d}\mathbf{\mathrm{x}}(t)}{\mathrm{d}t} = A\mathbf{\mathrm{x}}(t)+ B\mathbf{\mathrm{u}}(t)\text{。}
%\label{eq:control}
%\end{equation}
%
%其中:
%\begin{itemize}
%\item 向量$\mathbf{\mathrm{x}}(t)$表示$N$个在$t$时
%\item $A$表示$N$个度
%\item $\mathbf{\mathrm{u}}$
%\item $B$表示位点
%\end{itemize}
%
%\pause
%\begin{block}{注意}
%这是一个block。
%\end{block}
%\end{frame}
%
%%------------------------------------------------
%
%\begin{frame}
%\frametitle{Table}
%\begin{table}
%\begin{tabular}{l l l}
%\toprule
%\textbf{Treatments} & \textbf{Response 1} & \textbf{Response 2}\\
%\midrule
%Treatment 1 & 0.0003262 & 0.562 \\
%Treatment 2 & 0.0015681 & 0.910 \\
%Treatment 3 & 0.0009271 & 0.296 \\
%\bottomrule
%\end{tabular}
%\caption{Table caption}
%\end{table}
%\end{frame}
%
%
%\subsection{定理、定义}
%
%\begin{frame}
%\frametitle{Theorem}
%\begin{theorem}[勾股定理]
%$c^2 = a^2 + b^2$
%\end{theorem}
%\end{frame}
%
%\begin{frame}[fragile] % Need to use the fragile option when verbatim is used in the slide
%  \frametitle{Verbatim}
%  \begin{example}[Theorem Slide Code]
%    \begin{verbatim}
%    \begin{frame}
%      \frametitle{Theorem}
%      \begin{theorem}[勾股定理]
%	$c^2 = a^2 + b^2$
%      \end{theorem}
%    \end{frame}
%    \end{verbatim}
%  \end{example}
%\end{frame}
%
%\subsection{代码高亮}
%
%\begin{frame}[fragile]
%  \frametitle{代码高亮}
%C代码:
%\begin{minted}[mathescape,
%               linenos,     % displays the line numbers
%               frame=leftline,
%		baselinestretch=0.5]{csharp}
%  /*
%  这里可以显示公式:
%  $\pi=\lim_{n\to\infty}\frac{P_n}{d}$ where $P$ is the perimeter
%  of an $n$-sided regular polygon circumscribing a
%  circle of diameter $d$.
%  */
%  const double pi = 3.1415926535
%\end{minted}
%
%python代码:
%\begin{minted}[mathescape, linenos]{python}
%# Returns $\sum_{i=1}ˆ{n}i$
%# 多样注释格式和缩进,行码
%def sum_from_one_to(n):
%    r = range(1, n + 1)
%    return sum(r)
%\end{minted}
%\end{frame}
%
%\subsection{文献引用举例}
%\begin{frame}
%  \frametitle{宇宙大爆炸的定义}
%$x^2+y^2=z^2$\bcite{bcite1}
%\pause
%\begin{definition}
%  \point{宇宙大爆炸}:$(X_0,Y_0)$当且仅当$\forall \epsilon >0$\bcite{bcite2}。
%\end{definition}
%\end{frame}
%
%
%
%%----------------------------------------
%
%\subsection{左右分栏和图形动画} 
%
%\begin{frame}
%\begin{columns}[c] % The "c" option specifies centered vertical alignment while the "t" option is used for top vertical alignment
%\column{.45\textwidth} % Left column and width
%  \begin{figure}
%    \begin{center}
%    \tikzstyle{linktype1visible}= [visible on=<{4-}>]
%    \tikzstyle{linktype1previsible}= [visible on=<{1-3}>]
%    \tikzstyle{nodetype1visible}= [visible on=<{4-}>]
%    \tikzstyle{nodetype1previsible}= [visible on=<{1-3}>]
%    \tikzstyle{linktype2visible}= [visible on=<{5-}>]
%    \tikzstyle{scalestyle}= [scale=0.7]
%      \begin{tikzpicture}
%	\node[scalestyle, visible on=<{1,2,3,4}>, nodetype1pre] (N-1) at (0, 1) {1};
%	\node[scalestyle, visible on=<{5-}>, nodetype1] (N-1) at (0, 1) {1};
%	\foreach \name/\x in {2/2, 3/3, 4/4, 5/5}
%	{
%	  \node[scalestyle, nodetype1pre, nodetype1previsible] (N-\name) at (0, \x) {$\name$};
%	  \node[scalestyle, nodetype1, nodetype1visible] (N-\name) at (0, \x) {$\name$};
%	}
%	\foreach \name/\x/\y in {6/1/4, 7/2/5, 8/3/4, 9/2/3}
%	{
%	  \node[scalestyle, nodetype1pre, nodetype1previsible] (N-\name) at (\x, \y) {$\name$};
%	  \node[scalestyle, nodetype1, nodetype1visible] (N-\name) at (\x, \y) {$\name$};
%	}
%	\foreach \name/\x/\y in {10/-1/2, 11/-2/2}
%	{
%	  \node[scalestyle, nodetype1pre, nodetype1previsible] (N-\name) at (\x, \y) {$\name$};
%	  \node[scalestyle, nodetype1, nodetype1visible] (N-\name) at (\x, \y) {$\name$};
%	}
%	\foreach \from/\to in {1/2, 2/3, 3/4, 4/5, 6/7, 7/8, 8/9, 9/6}
%	{
%	  \draw[unlinktype1, linktype1previsible] (N-\from) -- (N-\to);
%	}
%	\draw[unlinktype1](N-4) -- (N-6);
%	\draw[unlinktype1](N-2) -- (N-10);
%	\path[unlinktype1] (N-5) edge [loop above] ();
%	\draw[linktype1pre, linktype1previsible] (N-10) .. controls + (up:0.5cm) .. (N-11);
%	\draw[linktype1pre, linktype1previsible] (N-11) .. controls + (down:0.5cm) .. (N-10);
%
%	\foreach \from/\to in {1/2, 2/3, 3/4, 4/5, 6/7, 7/8, 8/9, 9/6}
%	{
%	  \draw[linktype1, linktype1visible] (N-\from) -- (N-\to);
%	}
%	\draw[linktype1, linktype1visible] (N-10) .. controls + (up:0.5cm) .. (N-11);
%	\draw[linktype1, linktype1visible] (N-11) .. controls + (down:0.5cm) .. (N-10);
%	\draw[linktype2, linktype2visible] (0,-1) --  (0,0.7);
%      \end{tikzpicture}
%    \end{center}
%  \end{figure}
%\column{.45\textwidth} % Left column and width
% \begin{itemize}
%        \item<1-| alert@1>
%	  第一条说明
%        \item<4-| alert@4>
%	  第二条说明
%        \item<5-| alert@5>
%	  第三条说明
%        \item<8-| alert@8>
%	  第四条说明
% \end{itemize}
% \visible<2>{\fadetitle{这是一个透明标题}}
%\end{columns}
%\visible<6>{
%\fadetitle{\(\displaystyle
%  A=\left( 
%    \begin{array}{*{20}{rcccccccccl}}
%      0&0&0&0&0&0&0&0&0&0&0\\
%      a&0&0&0&0&0&0&0&0&0&0\\
%      0&b&0&0&0&0&0&0&0&c&0\\
%    \end{array}
%  \right)
%  \)
%}
%}
%\end{frame}
%%%------------------------------------------------
%%
%\begin{frame}
%\begin{figure}
%  \begin{center}
%    \tikzstyle{nodetype1visible}= [visible on=<{2-}>]
%    \tikzstyle{nodetype1previsible}= [visible on=<{1}>]
%    \tikzstyle{linktype2visible}= [visible on=<{3-}>]
%    \tikzstyle{linktype1visible}= [visible on=<{2-}>]
%    \tikzstyle{linktype1previsible}= [visible on=<{1}>]
%
%    \begin{minipage}[b]{0.45\linewidth}
%      \begin{tikzpicture}[scale=0.7]
%	%-------------------------------------
%	\node[unnodetype1,label=below left:$x_2$] (x2)  at (0,0) {};
%	\node[unnodetype1,label=above right:$x_1$] (x1)  at (2,1) {};
%	\node[nodetype1pre,nodetype1previsible,label=below right:$x_4$] (x4)  at (2,-1) {};
%	\node[nodetype1,nodetype1visible,label=below right:$x_4$] (x4)  at (2,-1) {};
%	\node[unnodetype1,label=below right:$x_3$] (x3)  at (4,0) {};
%	%-------------------------------------
%	\draw[-latex] (x1) -- (x2);
%	\draw[-latex] (x1) -- (x3);
%	\draw[linktype1pre,linktype1previsible] (x1) -- (x4);
%	\draw[linktype1,linktype1visible] (x1) -- (x4);
%	\draw[linktype2,linktype2visible] (2,3) node [right] {$u_1$} --  (x1);
%	\draw[linktype2,linktype2visible] (4,2) node [right] {$u_3$}-- (x3);
%	\draw[linktype2,linktype2visible] (0,2) node [left] {$u_2$}-- (x2);
%	%-------------------------------------
%    \end{tikzpicture}
%    \end{minipage}
%    \quad
%    \pause
%    \begin{minipage}[b]{0.45\linewidth}
%    \begin{tikzpicture}[scale=0.7]
%	%-------------------------------------
%	\node[nodetype1pre,nodetype1previsible,label=below left:$x_2$] (x2)  at (0,0) {};
%	\node[nodetype1,nodetype1visible,label=below left:$x_2$] (x2)  at (0,0) {};
%	\node[unnodetype1,label=above right:$x_1$] (x1)  at (2,1) {};
%	\node[unnodetype1,label=below left:$x_4$] (x4)  at (2,-1) {};
%	\node[unnodetype1,label=below right:$x_3$] (x3)  at (4,0) {};
%	%-------------------------------------
%	\draw[-latex] (x1) -- (x4);
%	\draw[-latex] (x1) -- (x3);
%	\draw[linktype1pre,linktype1previsible] (x1) -- (x2);
%	\draw[linktype1,linktype1visible] (x1) -- (x2);
%	\draw[linktype2, linktype2visible] (2,3) node [right] {$u_1$} --  (x1);
%	\draw[linktype2, linktype2visible] (4,2) node [right] {$u_3$}-- (x3);
%	\draw[linktype2, linktype2visible] (4,-2) node [right] {$u_4$}-- (x4);
%	%-------------------------------------
%    \end{tikzpicture}
%    \end{minipage}
%    \begin{minipage}[b]{0.45\linewidth}
%      \begin{tikzpicture}[scale=0.7]
%	%-------------------------------------
%	\node[unnodetype1,label=below left:$x_2$] (x2)  at (0,0) {};
%	\node[unnodetype1,label=above right:$x_1$] (x1)  at (2,1) {};
%	\node[unnodetype1,label=below left:$x_4$] (x4)  at (2,-1) {};
%	\node[nodetype1pre,nodetype1previsible,label=below right:$x_3$] (x3)  at (4,0) {};
%	\node[nodetype1,nodetype1visible,label=below right:$x_3$] (x3)  at (4,0) {};
%	%-------------------------------------
%	\draw[-latex] (x1) -- (x2);
%	\draw[-latex] (x1) -- (x4);
%	\draw[linktype1pre, linktype1previsible] (x1) -- (x3);
%	\draw[linktype1, linktype1visible] (x1) -- (x3);
%	\draw[linktype2,linktype2visible] (2,3) node [right] {$u_1$} --  (x1);
%	\draw[linktype2,linktype2visible] (4,-2) node [right] {$u_4$}-- (x4);
%	\draw[linktype2,linktype2visible] (0,2) node [left] {$u_2$}-- (x2);
%	%-------------------------------------
%	\node[unnodetype1,label=right:Original node] at (6.25,2) {};
%	\visible<2->{\node[nodetype1,nodetype1visible,label=right:Matched node] at (6.25,1) {};}
%	\draw[linktype1,linktype1visible] (5, 0) -- (6.5, 0) node [right] {Link type 1};
%	\draw[linktype2,line after snake=2mm, linktype2visible] (5,-1) --(6.5,-1) node [right] {Link type 2};
%    \end{tikzpicture}
%    \end{minipage}
%  \end{center}
%\end{figure}
%\end{frame}
%
%%------------------------------------------------
%
%\section{思维导图} 
%\begin{frame}
%  \frametitle{思维导图,总结,致谢}
%  \setbeamercovered{invisible}
%\begin{tikzpicture}[scale=0.6, transform shape]
%\bf
%\centering
%  \path[mindmap,concept color=structure.fg,text=white]
%    node[concept] {Beamer模板}
%    [clockwise from=0]
%    child[concept color=green!30!black, visible on=<{2-}>] {
%      node[concept] {tikz作图}
%      [clockwise from=90]
%      child { node[concept] {节点} }
%      child { node[concept] {边} }
%      child { node[concept] {样式定义} }
%      child { node[concept] {样式分配} }
%    }
%    child[concept color=blue!80!black, visible on=<{3-}>] {
%      node[concept] {代码高亮}
%      [clockwise from=-30]
%      child { node[concept] {支持所有代码} }
%      child { node[concept] {支持公式} }
%    }
%    child[concept color=red!70!structure.fg, visible on=<{4-}>] { 
%      node[concept] {文献} 
%      [clockwise from=-30]
%      child { node[concept] {格式多样} }
%      child { node[concept] {引用方便} }
%    }
%    child[concept color=orange!70!structure.fg, visible on=<{5-}>] { 
%      node[concept] {主题} 
%      [clockwise from=-90]
%      child { node[concept] {主题颜色} }
%      child { node[concept] {背景和前景} }
%      child { node[concept] {字体} }
%      child { node[concept] {图形} }
%    };
%\end{tikzpicture}
%\visible<6->{\fadetitle{\Huge{\centerline{谢谢}}}}
%\end{frame}
%
%%------------------------------------------------

